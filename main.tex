\documentclass[review]{elsarticle}

\usepackage{lineno,hyperref}
\usepackage[brazilian]{babel}
\usepackage[utf8x]{inputenc}

\modulolinenumbers[5]

%\journal{}

%%%%%%%%%%%%%%%%%%%%%%%
%% Elsevier bibliography styles
%%%%%%%%%%%%%%%%%%%%%%%
%% To change the style, put a % in front of the second line of the current style and
%% remove the % from the second line of the style you would like to use.
%%%%%%%%%%%%%%%%%%%%%%%

%% Numbered
%\bibliographystyle{model1-num-names}

%% Numbered without titles
%\bibliographystyle{model1a-num-names}

%% Harvard
%\bibliographystyle{model2-names.bst}\biboptions{authoryear}

%% Vancouver numbered
%\usepackage{numcompress}\bibliographystyle{model3-num-names}

%% Vancouver name/year
%\usepackage{numcompress}\bibliographystyle{model4-names}\biboptions{authoryear}

%% APA style
%\bibliographystyle{model5-names}\biboptions{authoryear}

%% AMA style
%\usepackage{numcompress}\bibliographystyle{model6-num-names}

%% `Elsevier LaTeX' style
\bibliographystyle{elsarticle-num}
%%%%%%%%%%%%%%%%%%%%%%%

\begin{document}

\begin{frontmatter}

\title{Ataques de Canal Lateral Contra Enclaves Intel SGX}
%\tnotetext[mytitlenote]{Fully documented templates are available in the elsarticle package on \href{http://www.ctan.org/tex-archive/macros/latex/contrib/elsarticle}{CTAN}.}

%% Group authors per affiliation:
\author{Fábio Silva}
\address{Laboratório de Sistemas Distribuídos\\Universidade Federal de Campina Grande}
\ead{fabiosilva@lsd.ufcg.edu.br}
%\fntext[myfootnote]{Since 1880.}

%% or include affiliations in footnotes:
%\author[mymainaddress,mysecondaryaddress]{Elsevier Inc}
%\ead[url]{www.elsevier.com}

%\author[mysecondaryaddress]{Global Customer Service\corref{mycorrespondingauthor}}
%\cortext[mycorrespondingauthor]{Corresponding author}
%\ead{support@elsevier.com}

%\address[mymainaddress]{1600 John F Kennedy Boulevard, Philadelphia}
%\address[mysecondaryaddress]{360 Park Avenue South, New York}

\begin{abstract}
O isolamento e a proteção de processos em nível de usuário em máquinas compartilhadas têm sido tradicionalmente o domínio do sistema operacional ou do hipervisor. No entanto, em um ambiente de computação na nuvem, os usuários não podem confiar no software do sistema para prover isolamento, uma vez que o provedor de serviços responsável por manter o software privilegiado pode não ser confiável. O Intel Software Guard eXtensions (SGX) é uma extensão de segurança baseada em hardware que permite aplicações em nível de usuário serem executadas de forma segura em um ambiente onde todos os outros softwares em execução no sistema não são confiáveis. Intel SGX usa enclaves seguros em execução na memória protegida, juntamente com um esquema de atestação de software para fornecer garantias de confidencialidade e integridade aos usuários que desejam executar o software em um sistema remoto não confiável. O SGX é vulnerável a vários ataques de canal lateral baseados em software, que aproveitam as medições de desempenho para determinar padrões de acesso à memória e derivar segredos do software executado em enclaves seguros do SGX. Esta revisão sistemática fornece uma visão geral das vulnerabilidades do SGX para ataques de canal lateral baseados em software, bem como possíveis contramedidas que podem ser implementadas para ajudar a proteger os programas SGX existentes.
\end{abstract}

\begin{keyword}
\texttt Intel SGX\sep ataques de canal lateral \sep computação na nuvem
%\MSC[2010] 00-01\sep  99-00
\end{keyword}

\end{frontmatter}

\linenumbers

\section{Introdução}

Os serviços de computação em nuvem, como o EC-2 da Amazon e o Google Cloud Platform, registraram um crescimento significativo, à medida que os avanços nos sistemas distribuídos tornam a utilização de recursos computacionais de terceiros mais atraente e acessível ~\cite{lunden_amazons_2015}. Como resultado, a segurança em uma configuração de nuvem tomou-se cada vez mais relevante à medida que os desenvolvedores procuram maneiras de proteger dados confidenciais de partes potencialmente não confiáveis com quem compartilham recursos de computação. Os usuários têm tradicionalmente confiado nos serviços oferecidos pelo sistema operacional para fornecer garantias de confidencialidade e integridade, exercendo controle estrito sobre o gerenciamento de recursos e impondo o isolamento lógico dos processos em nível de usuário. No entanto, em um cenário de computação em nuvem, os usuários que desejam executar o software em sistemas remotos podem não confiar no provedor de serviços em nuvem responsável pela administração do sistema operacional ou do hipervisor do host. Portanto, é necessário considerar um cenário em que um usuário deseja executar com segurança seu software em um sistema controlado por um provedor de serviços potencialmente mal-intencionado, com controle sobre o software privilegiado.

O Intel Software Guard eXtensions visa resolver esse problema fornecendo um Ambiente de Execução Confiável por meio da adição de extensões de hardware que fornecem fortes garantias de confidencialidade e integridade em um ambiente em que nenhum software em uma máquina host é confiável. O SGX reforça o isolamento estrito do software protegido em nível de usuário do resto do sistema não confiável e usa um esquema de atestação de software para provar ao usuário que seu software é não modificado e protegido por hardware seguro ~\cite{intel_corporation_intel_2016}.

Embora o SGX forneça fortes proteções contra os adversários que tentam acessar ou manipular diretamente softwares protegidos, o SGX não protege contra vários ataques de canal lateral baseados em software, que usam informações derivadas de detalhes de implementação para permitir que um atacante deduza segredos de softwares protegidos ~\cite{costan_intel_2016}.

Este artigo de revisão sistemática fornece uma visão geral, sem viés, do modelo de design e segurança do SGX e detalha vários ataques de canal lateral distintos aos quais o SGX é vulnerável. Além disso, as contramedidas baseadas em software e hardware são examinadas, e as implicações deste trabalho para pesquisas futuras e o desenvolvimento de novos Ambientes de Execução Confiáveis são discutidos.

É provida uma evidência empírica com foco na seguinte \textbf{questão de pesquisa (RQ)}: quais ataques de canal lateral podem ser realizados contra enclaves Intel SGX e como mitigá-los?



\paragraph{Installation} If the document class \emph{elsarticle} is not available on your computer, you can download and install the system package \emph{texlive-publishers} (Linux) or install the \LaTeX\ package \emph{elsarticle} using the package manager of your \TeX\ installation, which is typically \TeX\ Live or Mik\TeX.

\paragraph{Usage} Once the package is properly installed, you can use the document class \emph{elsarticle} to create a manuscript. Please make sure that your manuscript follows the guidelines in the Guide for Authors of the relevant journal. It is not necessary to typeset your manuscript in exactly the same way as an article, unless you are submitting to a camera-ready copy (CRC) journal.

\paragraph{Functionality} The Elsevier article class is based on the standard article class and supports almost all of the functionality of that class. In addition, it features commands and options to format the
\begin{itemize}
\item document style
\item baselineskip
\item front matter
\item keywords and MSC codes
\item theorems, definitions and proofs
\item lables of enumerations
\item citation style and labeling.
\end{itemize}

\section{Front matter}

The author names and affiliations could be formatted in two ways:
\begin{enumerate}[(1)]
\item Group the authors per affiliation.
\item Use footnotes to indicate the affiliations.
\end{enumerate}
See the front matter of this document for examples. You are recommended to conform your choice to the journal you are submitting to.

\section{Bibliography styles}

There are various bibliography styles available. You can select the style of your choice in the preamble of this document. These styles are Elsevier styles based on standard styles like Harvard and Vancouver. Please use Bib\TeX\ to generate your bibliography and include DOIs whenever available.

Here are two sample references: \cite{Feynman1963118,Dirac1953888}.

%\section*{References}

\bibliography{mybibfile}

\end{document}