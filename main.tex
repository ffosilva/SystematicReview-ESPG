\documentclass[review]{elsarticle}

\usepackage{lineno,hyperref}
\usepackage[brazilian]{babel}
\usepackage[utf8x]{inputenc}

\usepackage[T1]{fontenc}
%\usepackage{enumitem}

%% Useful packages
\usepackage{placeins}
\usepackage{lineno,hyperref}
\usepackage[table]{xcolor}
\usepackage{multirow}
\usepackage{amssymb}
\usepackage{supertabular}
\usepackage{footnote}
\makesavenoteenv{tabular}
\makesavenoteenv{table}
\modulolinenumbers[5]
\usepackage{hyperref}

\modulolinenumbers[5]

%\journal{}

%%%%%%%%%%%%%%%%%%%%%%%
%% Elsevier bibliography styles
%%%%%%%%%%%%%%%%%%%%%%%
%% To change the style, put a % in front of the second line of the current style and
%% remove the % from the second line of the style you would like to use.
%%%%%%%%%%%%%%%%%%%%%%%

%% Numbered
%\bibliographystyle{model1-num-names}

%% Numbered without titles
%\bibliographystyle{model1a-num-names}

%% Harvard
%\bibliographystyle{model2-names.bst}\biboptions{authoryear}

%% Vancouver numbered
%\usepackage{numcompress}\bibliographystyle{model3-num-names}

%% Vancouver name/year
%\usepackage{numcompress}\bibliographystyle{model4-names}\biboptions{authoryear}

%% APA style
%\bibliographystyle{model5-names}\biboptions{authoryear}

%% AMA style
%\usepackage{numcompress}\bibliographystyle{model6-num-names}

%% `Elsevier LaTeX' style
\bibliographystyle{elsarticle-num}
%%%%%%%%%%%%%%%%%%%%%%%

\begin{document}

\begin{frontmatter}

\title{Ataques de Canal Lateral Contra Enclaves Intel SGX}
%\tnotetext[mytitlenote]{Fully documented templates are available in the elsarticle package on \href{http://www.ctan.org/tex-archive/macros/latex/contrib/elsarticle}{CTAN}.}

%% Group authors per affiliation:
\author{Fábio Silva}
\address{Laboratório de Sistemas Distribuídos\\Universidade Federal de Campina Grande}
\ead{fabiosilva@lsd.ufcg.edu.br}
%\fntext[myfootnote]{Since 1880.}

%% or include affiliations in footnotes:
%\author[mymainaddress,mysecondaryaddress]{Elsevier Inc}
%\ead[url]{www.elsevier.com}

%\author[mysecondaryaddress]{Global Customer Service\corref{mycorrespondingauthor}}
%\cortext[mycorrespondingauthor]{Corresponding author}
%\ead{support@elsevier.com}

%\address[mymainaddress]{1600 John F Kennedy Boulevard, Philadelphia}
%\address[mysecondaryaddress]{360 Park Avenue South, New York}

\begin{abstract}
O isolamento e a proteção de processos em nível de usuário em máquinas compartilhadas têm sido tradicionalmente o domínio do sistema operacional ou do hipervisor. No entanto, em um ambiente de computação na nuvem, os usuários não podem confiar no software do sistema para prover isolamento, uma vez que o provedor de serviços responsável por manter o software privilegiado pode não ser confiável. O Intel Software Guard eXtensions (SGX) é uma extensão de segurança baseada em hardware que permite aplicações em nível de usuário serem executadas de forma segura em um ambiente onde todos os outros softwares em execução no sistema não são confiáveis. Intel SGX usa enclaves seguros em execução na memória protegida, juntamente com um esquema de atestação de software para fornecer garantias de confidencialidade e integridade aos usuários que desejam executar o software em um sistema remoto não confiável. O SGX é vulnerável a vários ataques de canal lateral baseados em software, que aproveitam as medições de desempenho para determinar padrões de acesso à memória e derivar segredos do software executado em enclaves seguros do SGX. Esta revisão sistemática fornece uma visão geral das vulnerabilidades do SGX para ataques de canal lateral baseados em software.
\end{abstract}

\begin{keyword}
\texttt Intel SGX\sep ataques de canal lateral \sep computação na nuvem
%\MSC[2010] 00-01\sep  99-00
\end{keyword}

\end{frontmatter}

\linenumbers

\section{Introdução}

Os serviços de computação em nuvem, como o EC-2 da Amazon e o Google Cloud Platform, registraram um crescimento significativo, à medida que os avanços nos sistemas distribuídos tornam a utilização de recursos computacionais de terceiros mais atraente e acessível ~\cite{lunden_amazons_2015}. Como resultado, a segurança em uma configuração de nuvem tomou-se cada vez mais relevante à medida que os desenvolvedores procuram maneiras de proteger dados confidenciais de partes potencialmente não confiáveis com quem compartilham recursos de computação. Os usuários têm tradicionalmente confiado nos serviços oferecidos pelo sistema operacional para fornecer garantias de confidencialidade e integridade, exercendo controle estrito sobre o gerenciamento de recursos e impondo o isolamento lógico dos processos em nível de usuário. No entanto, em um cenário de computação em nuvem, os usuários que desejam executar o software em sistemas remotos podem não confiar no provedor de serviços em nuvem responsável pela administração do sistema operacional ou do hipervisor do host. Portanto, é necessário considerar um cenário em que um usuário deseja executar com segurança seu software em um sistema controlado por um provedor de serviços potencialmente mal-intencionado, com controle sobre o software privilegiado.

O Intel Software Guard eXtensions visa resolver esse problema fornecendo um Ambiente de Execução Confiável por meio da adição de extensões de hardware que fornecem fortes garantias de confidencialidade e integridade em um ambiente em que nenhum software em uma máquina host é confiável. O SGX reforça o isolamento estrito do software protegido em nível de usuário do resto do sistema não confiável e usa um esquema de atestação de software para provar ao usuário que seu software é não modificado e protegido por hardware seguro ~\cite{intel_corporation_intel_2016}.

Embora o SGX forneça fortes proteções contra os adversários que tentam acessar ou manipular diretamente softwares protegidos, o SGX não protege contra vários ataques de canal lateral baseados em software, que usam informações derivadas de detalhes de implementação para permitir que um atacante deduza segredos de softwares protegidos ~\cite{costan_intel_2016}.

Este artigo de revisão sistemática fornece uma visão geral, sem viés, do modelo de design e segurança do SGX e detalha vários ataques de canal lateral distintos aos quais o SGX é vulnerável. Além disso, as contramedidas baseadas em software e hardware são examinadas. É provida uma evidência empírica com foco na seguinte \textbf{questão de pesquisa (RQ)}: Quais ataques de canal lateral podem ser realizados contra enclaves Intel SGX?

\section{Ataques de Canal Lateral de Software}

Os ataques de canal lateral são uma classe de ataques que aproveitam informações sobre as propriedades físicas de um sistema para realizar um ataque. Ataques de canal lateral físico usam o conhecimento das características de um sistema para inferir segredos de vazamentos de informações físicas. Por exemplo, um atacante pode analisar o consumo de energia ou a radiação eletromagnética para derivar segredos de uma máquina que executa cálculos de alguma forma seguros ~\cite{standaert_introduction_2010}.

Os ataques de canal lateral de software exploram vazamentos de informações que resultam de características das implementações de hardware ou software de um sistema seguro que podem ser observados pelo software em execução no próprio sistema. Por exemplo, um atacante pode aproveitar a diferença de tempo necessária para acessar dados armazenados no cache em comparação com os dados armazenados na memória principal para inferir segredos dos padrões de uso de memória ~\cite{costan_intel_2016}.

Embora os ataques físicos de canal lateral contra hardware moderno sejam difíceis e caros, exigindo ferramentas avançadas e acesso físico à máquina vítima, os ataques de canal lateral de software são baratos de implantar e podem ser executados por qualquer pessoa com acesso remoto ou local a um sistema ~\cite{costan_intel_2016}. Os ataques de canal lateral de software discutidos aqui exploram detalhes de implementação de hardware e software para obter informações sobre padrões de acesso à memória, que podem ser usados para inferir segredos de um sistema que, de alguma forma, seria seguro ~\cite{gotzfried_cache_2017, schwarz_malware_2017, xu_controlled-channel_2015, shinde_preventing_2015}. Os ataques de canal lateral não são incluídos no modelo de ameaça SGX devido à impraticabilidade percebida dessa classe de ataques. No entanto, como o trabalho analisado abaixo demonstra, os ataques de canal lateral de software são práticos contra enclaves SGX e representam uma ameaça significativa à segurança do programa em um ambiente de nuvem no qual o software privilegiado não é confiável ~\cite{moghimi_cachezoom:_2017, intel_corporation_tutorial_2015}.

\section{Ataques Contra Enclaves SGX na Literatura}

Para responder quais ataques de canal lateral podem ser realizados contra enclaves Intel SGX \textbf{(RQ)} foi conduzida uma revisão na literatura.

\subsection {Seleção dos artigos}

Para seleção dos artigos realizou-se, primeiramente, a leitura dos resumos das publicações selecionadas, através da plataforma \emph{Google Scholar} com o objetivo de refinar a amostra por meio de critérios de inclusão e exclusão. Foram incluídos artigos originais publicados entre 2007 e 2017. Para escolha inicial dos artigos, foi utilizada a seguinte \emph{string} de busca genérica: ("intel sgx") and ("side-channel"). Por meio desse processo, foi delimitado quais tipos de ataques de canal o Intel SGX era vulnerável. Delimitados os tipos de ataques, foram feitas mais três buscas, uma para cada tipo de ataque. As \emph{strings} de busca foram: ("intel sgx") and ("side-channel") and (("cache attacks") or ("page faults") or ("timing attacks") or ("branch prediction")). Foram escolhidos os 10 primeiros artigos, ordenados pela relevância.

\section{Ataques Contra Enclaves SGX}

Vários ataques distintos de canal lateral de software foram demonstrados contra enclaves SGX em ataques de prova de conceito executados por atacantes privilegiados e de nível de usuário. Embora os dados do enclave estejam protegidos na Memória Reservada do Processador e sejam isolados dos processos do SO e do usuário, o gerenciamento e o escalonamento da memória são delegados para o SO e/ou hipervisor não confiável, oferecendo ao SO um grande grau de controle sobre o processo SGX vítima e fornecendo recursos de medição de granulação fina. Assim, um SO mal-intencionado é livre para ditar com precisão como e quando um programa de enclave vítima é escalonado e em quais núcleos lógicos. Essa alta magnitude de controle sobre o processo vítima pode ser aproveitada para realizar poderosos ataques de canal lateral de software privilegiado contra enclaves SGX ~\cite{costan_intel_2016}.

Além disso, as garantias de proteção e isolamento fornecidas pelo SGX se estendem a programas maliciosos, o que significa que um adversário pode usar um enclave SGX para lançar ataques contra um enclave vítima em execução no mesmo sistema host. Como até mesmo os serviços privilegiados desconhecem o conteúdo e o comportamento dos enclaves SGX, o sistema operacional host não tem meios de monitorar seus enclaves SGX quanto a comportamento mal-intencionado e, portanto, não protege processos honestos de usuários contra comportamentos de enclaves mal-intencionados ~\cite{costan_intel_2016, schwarz_malware_2017}.

\subsection{Ataques de Temporização}

Caches são pequenas regiões de memória de alta velocidade usadas para armazenar dados acessados recentemente, a fim de reduzir tempos de acesso futuros. Em computadores modernos com vários níveis de cache, cada núcleo normalmente possui seu primeiro e segundo caches (L1 e L2), com um cache de terceiro nível (L3), também chamado de cache de nível mais baixo (LLC), compartilhado entre todos os núcleos. Quando a CPU precisa acessar dados, os acessos podem ser resolvidos em um cache (um cache ``hit''), ou os dados podem precisar ser carregados da memória principal (um cache ``miss''), neste ponto então sendo armazenandos no cache, expulsando dados mais antigos. Os acessos de cache são consideravelmente mais rápidos que os acessos que devem ser resolvidos na memória principal, e assim os dados armazenados em cache podem reduzir significativamente o tempo de computação, reduzindo a dependência de acessos de memória principal relativamente lentos. Os ataques de temporização de cache exploram as diferenças de tempo entre acertos e erros do cache para inferir segredos mantidos por um programa em execução concorrente, analisando os padrões de acesso à memória dos processos da vítima ~\cite{costan_intel_2016, moghimi_cachezoom:_2017}.

Em alto nível, um atacante pode aproveitar essa diferença na velocidade de acesso preenchendo um cache compartilhado com dados, despejando assim dados usados por um processo vítima. O atacante então mede o tempo necessário para acessar esses dados depois que o processo vítima tiver a chance de ser executado. Blocos de memória que demoram mais para serem acessados provavelmente foram despejados do cache para liberar espaço para os dados usados pelo processo vítima, o que significa que a memória acessada do processo vítima foi mapeada para a mesma região do cache compartilhado. Como os blocos de memória são mapeados para regiões de armazenamento em cache com base no endereço de memória virtual, o atacante pode usar um ataque de cache para obter informações sobre os padrões de uso de memória do processo vítima, que podem ser usados para extrair segredos como chaves de criptografia ~\cite{liu_cache-timing_2013}. Dessa forma, o atacante pode inferir quais regiões da memória foram acessadas pelo processo vítima com base no tempo necessário para acessar os dados que ele havia colocado anteriormente no cache e usar essas informações para derivar os segredos do programa e comprometer a confidencialidade ~\cite{moghimi_cachezoom:_2017, gotzfried_cache_2017}.

O ataque \emph{Prime+Probe} é uma forma de ataque de cache que pode ser usada por adversários privilegiados e não-privilegiados para atacar um processo vítima, tendo como alvo um cache compartilhado. Normalmente, esse ataque é usado para direcionar o cache L1 em um núcleo \emph{multithreaded} compartilhado entre o processo do atacante e o processo vítima, embora ele também possa ser adaptado para uso na LLC (cache). Em uma etapa \emph{priming}, o atacante preenche todo o cache compartilhado, fazendo com que os dados em cache da vítima sejam despejados. Em seguida, o atacante aguarda a execução do processo da vítima e, nesse momento, ele faz os acessos à memória que resultam em blocos de memória do atacante sendo despejados do cache compartilhado. Finalmente, em uma etapa \emph{probe}, o atacante tenta acessar seus dados, medindo o tempo de acesso à memória para determinar quais blocos de memória foram usados pelo processo vítima ~\cite{moghimi_cachezoom:_2017, gotzfried_cache_2017}.

Provas de conceito de ataques contra várias implementações do AES demonstraram que um atacante pode lançar poderosos ataques de cache Prime+Probe contra enclaves SGX, aproveitando os privilégios de escalonamento no nivel do sistema operacional e as ferramentas de monitoramento de desempenho. Ao obter o controle sobre um sistema operacional host mal-intencionado para escalonar um programa de ataque Prime+Probe no mesmo núcleo do enclave vítima com todos os outros softwares isolados para outros núcleos, um atacante pode eliminar o ruído no cache causado pela execução de outros processos. Além disso, como o atacante também pode controlar como ocorre o escalonamento, o processo do atacante pode interromper o processo vítima com frequência, fornecendo uma resolução temporal muito boa, ou seja, reduzindo o tempo de CPU permitido à vítima para que o atacante possa investigar \emph{cache misses}. O atacante pode fazer inferências mais fortes sobre o tempo e a ordem dos acessos à memória pelo processo da vítima. Em ataques de prova de conceito, um atacante pode derivar uma chave secreta de implementações modernas do AES em apenas 10 observações. Assim, um atacante com controle sobre um SO mal-intencionado pode obter o controle sobre decisões de planejamento para realizar poderosos ataques de cache de alta resolução capazes de derivar chaves secretas de enclaves vítimas executando rotinas criptográficas sem precisar contornar diretamente as proteções oferecidas pelo SGX ~\cite{moghimi_cachezoom:_2017, gotzfried_cache_2017}.

Além disso. um atacante remoto pode aproveitar as proteções oferecidas pelos enclaves SGX para ocultar códigos maliciosos e realizar ataques de cache contra enclaves vítimas co-localizados usando apenas recursos disponiveis para processos no nível do usuário. Nesse caso, o atacante não tem controle sobre onde ou quando os processos do atacante e vítima serão escalonados e, portanto, o atacante não pode aproveitar as decisões de sincronização ou escalonamento para melhorar o ruido ou a resolução das informações que vazaram. No entanto, essa abordagem oferece ao atacante as garantias de confidencialidade oferecidas pelo SGX, o que significa que esse é um mecanismo eficaz para ocultar ataques contra outros enclaves. Foi demonstrado que esse ataque é capaz de extrair chaves privadas de um enclave executando uma implementação RSA atual após apenas 5 minutos de observação ~\cite{schwarz_malware_2017}. Isso demonstra que os enclaves SGX não são apenas vulneráveis a ataques lançados por um adversário privilegiado, aproveitando o poder de um SO mal-intencionado, mas também a ataques realizados por adversários remotos que podem usar as garantias de confidencialidade oferecidas pelo SGX para se distinguirem de usuários honestos.

\subsection{Ataques de Falha de Página}

Semelhante aos ataques de temporização, os ataques de falha de página de \emph{canal controlado} permitem que um atacante obtenha segredos de um programa de vítima baseado em padrões de acesso à memória. No entanto, ao contrário dos ataques de temporização de cache, ataques de falha de página de canal controlado dependem de um atacante privilegiado que aproveita o controle sobre eventos do sistema para forçar um processo de vítima a gerar falhas de página, revelando informações sobre seus padrões de acesso à memória que podem ser usados para inferir segredos e comprometer a confidencialidade ~\cite{xu_controlled-channel_2015}.

Os ataques de falha de página do canal controlado dependem de um atacante ter controle sobre o gerenciamento de memória, como é o caso quando um atacante controla um sistema operacional malicioso executado em um sistema no qual a vítima está executando um código em um enclave SGX ~\cite{intel_corporation_tutorial_2015}. Para realizar um ataque, é necessario primeiro analisar o código-fonte da vítima ou o executável binário para identificar acessos a dados dependentes de entrada e transferências de controle dependentes de entrada, o que geralmente é possível, já que muitas rotas criptográficas são de código aberto. Quando o aplicativo é executado, o atacante pode restringir o acesso a um conjunto de destino das páginas da vítima, fazendo com que uma falha de página seja acionada sempre que a vítima tentar acessar um endereço em uma página de destino. Como a CPU deve reportar pelo menos o endereço base da página com falha ao sistema operacional, o atacante pode registrar sequências de acessos à página e usar essas informações para executar uma análise off-line e fazer inferências sobre os dados secretos da vítima com granularidade no nível da página (4 KB) ~\cite{xu_controlled-channel_2015, shinde_preventing_2015}.

Ataques de falha de página de canal controlado são viáveis contra enclaves SGX desde que um atacante privilegiado tenha controle sobre gerenciamento de memória, e o SGX tem como objetivo dar suporte a aplicativos legados sem exigir que eles sejam modificados para ocultar dados privados de um atacante que possa observar os padrões de acesso à memória ~\cite{xu_controlled-channel_2015}. Além disso, um SO mal-intencionado pode exercer um controle refinado sobre a alocação de páginas para executar um ataque de ``casa dos pombos'' contra um enclave SGX. Neste caso, o sistema operacional aloca apenas 3 páginas ao enclave da vítima - uma página de código, uma página de origem e uma página de destino - que compõem o ``conjunto de casa dos pombos''. Qualquer acesso à memória que esteja fora dessas páginas causará uma falha na página, revelando ao sistema operacional quais dados ou códigos o aplicativo está tentando acessar. Ao causar repetidamente que a vítima gere falhas de página e ajuste o conjunto de casa dos pombos, o atacante pode observar atentamente a sequência de acessos feitos pela vítima, maximizando a informação que vazou para o atacante através do canal lateral de falha de página. Esta técnica foi demonstrada contra enclaves SGX executando implementações de OpenSSL e Libgcrypt ~\cite{shinde_preventing_2015}.

O alto grau de controle do sistema operacional sobre os eventos do sistema possibilitam esses poderosos e silenciosos ataques de canal lateral, que não podem ser realizados em um ambiente de computação tradicional em que o sistema operacional é confiável. Embora o modelo de ameaças do SGX exclua explicitamente ataques de canal lateral, a praticidade desses mecanismos como vetores de ataque não pode ser ignorada, já que o SGX permite que o SO exerça controle total sobre o gerenciamento de memória e o escalonamento.

\subsection{Análise de Pevisão de Ramificação}

Os processadores modernos usam \emph{pipelining} de instrução para buscar e decodificar instruções enquanto instruções anteriores estão sendo executadas. Para manter a eficiência quando uma ramificação é encontrada, o processador prevê qual instrução será executada após a ramificação com base no comportamento anterior. A próxima instrução prevista é buscada antes do processador determinar qual ramificação deve ser executada e, se a previsão estiver correta, a execução em pipeline pode continuar normalmente. Em um desvio de ramo, o processador deve limpar a instrução prevista e carregar a instrução correta antes que a execução possa continuar. Em ambos os casos, o resultado é usado como base para a previsão na próxima vez que o ramo é encontrado ~\cite{aciicmez_power_2007, lee_inferring_2017}.

Um atacante privilegiado pode aproveitar esse comportamento para inferir informações sobre um processo de vítima por meio de um ataque de \emph{sombreamento de ramificação}. Resumidamente, um atacante usa o conhecimento da estrutura de um programa para construir um programa ``sombra'', que possui ramificações nos mesmos pontos em execução do processo vítima. O atacante pode interromper o processo vítima depois que uma ramificação for encontrada e executar o programa ``sombra'' iniciando no endereço da ramificação mais recentemente encontrada no programa da vítima. O atacante pode então usar diferenças temporais ou ferramentas de monitoramento de desempenho para determinar se as predições de ramificação feitas pelo processador com base no comportamento da vítima levaram a previsões corretas ou incorretas do comportamento do programa de sombra e podem inferir informações sobre os padrões de execução do processo vítima ~\cite{lee_inferring_2017}.

O SGX não limpa o histórico de ramificações de um enclave ao executar uma saída de enclave e um atacante privilegiado pode tirar proveito disso, forçando interrupções frequentes, de modo que todas as ramificações tomadas pela vítima possam ser determinadas pelo atacante. A fim de reduzir o ruído inerente na tentativa de diferenças temporais no desempenho entre predições de ramificação corretas e incorretas no programa ``sombra'', um atacante pode fazer uso do Last Branch Record (LBR), um recurso de depuração da CPU que registra a última ramificação tomada e se a previsão da CPU estava correta. Esse recurso só está disponível no modo de depuração, portanto, o atacante deve configurar o programa de sombra para ser executado no modo de depuração e verificar se as previsões baseadas no comportamento da vítima levaram a escolhas corretas no programa de sombra. O atacante pode, portanto, usar esse mecanismo de depuração privilegiado para determinar com precisão quais ramificações foram tomadas pela vítima e inferir informações sobre a entrada para o programa da vítima. Ataques de prova de conceito demonstraram que essa técnica pode revelar informações sobre o comportamento de entrada e de programa protegidos em enclaves SGX que não podem ser detectados pelos ataques de falha de página de canal controlado descritos acima ~\cite{lee_inferring_2017, shinde_preventing_2015}. Aqui, um atacante pode fazer uso dos mecanismos de depuração disponibilizados pelo hardware para o software privilegiado, a fim de eliminar o ruído em um canal lateral impraticável. Embora o sombreamento de ramificação não tenha sido demonstrado como um mecanismo viável para inferir uma chave secreta, esses ataques são capazes de derivar informações potencialmente confidenciais onde outros ataques de canal lateral falham.

\section{Conclusões}

Enclaves Intel SGX são vulneráveis a vário ataques de canal lateral de software que podem ser implementados de maneira barata e fácil por um atacante com acesso remoto ou local à máquina host. Os atacantes privilegiados podem aproveitar o controle sobre o escalonamento, o gerenciamento de memória e o acesso a ferramentas poderosas de monitoramento de desempenho para realizar poderosos ataques de canal lateral. Além disso, os atacantes sem privilégios podem usar as proteções oferecidas pelos enclaves para disfarçar ataques direcionados a outros enclaves em execução no mesmo sistema host. Essas descobertas são especialmente preocupantes em um ambiente de nuvem, em que o administrador de um sistema remoto pode não ser confiável, e usuários mal-intencionados podem ocultar seu comportamento da supervisão do sistema operacional, do software antivírus ou do administrador do sistema. Embora todos os ataques descritos aqui sejam ataques de prova de conceito, eles poderiam ser implementados em sistemas em produção, já que os sistemas típicos de computação em nuvem são distribuídos por natureza e um atacante privilegiado é livre para redistribuir o tráfego para eliminar ruídos e isolar a vítima.

Para fornecer garantias de segurança aos usuários que executam programas em um ambiente de nuvem, o SGX e outros ambientes de execução confiáveis devem levar em consideração os ataques de canal lateral e impedir que um atacante privilegiado deduza segredos por meio de mecanismos de monitoramento ou depuração de desempenho. A dificuldade em fornecer essas garantias reside na ampla superfície de ataque que o controle sobre um sistema operacional malicioso oferece a um atacante, embora várias soluções baseadas em software, compilador e hardware apresentem técnicas promissoras além da segurança existente oferecida pela SGX.

\bibliography{mybibfile}

\end{document}
